\documentclass[12pt,letterpaper]{article}
\usepackage[utf8]{inputenc}
\usepackage{amsmath}
\usepackage{amsfonts}
\usepackage{amssymb}
\usepackage{amsthm}  
\usepackage[top=1in, bottom=1in, left=1in, right=1in]{geometry}
\usepackage{titlesec}
\usepackage[backend=bibtex]{biblatex}
\usepackage{hyperref}
\usepackage[]{footmisc}
\usepackage{mathrsfs}
\usepackage{graphicx}
\usepackage{verbatim}
\usepackage{algpseudocode}
\usepackage{textcomp}
\usepackage{color}
\usepackage[table]{xcolor}
\titleformat{\section}{\normalsize\scshape\center}{\thesection}{1em}{}
\titleformat{\subsection}{\normalsize\scshape\center}{\thesubsection}{1em}{}
\makeatletter
\renewcommand\@makefntext[1]{%
    \parindent 1em%
    \@thefnmark.~#1}
\makeatother
\begin{document}
\title{\uppercase{\textbf{\normalsize A Knowledge Database}}}
\author{\small{\textsc{Lukas Nabergall}}}
\date{\small{\textsc{\today}}}
\maketitle

\section{Principles}

This knowledge database application is designed with the following principles in mind:
\begin{enumerate}
\item[1.] Should contain content which rapidly converges in time towards validated knowledge. 
\item[2.] Should be relevant and useful to advanced research, development, and academia.
\item[3.] Should optimize searchability and quick content recognition and absorption.
\item[4.] Should be based upon open content submission, editing, and validation.
\end{enumerate}


\section{Users}

The knowledge database will be completely open to all for viewing and contributing, but users will have the option to create an account and, for transparency, will be encouraged to provide their full name. Logged in users will be able to create new content, have the option to be denoted as authors of pieces of content that they have created or edited, and will be able to validate new edits on all pieces of content of which they are authors. Users who are not logged into an account, or who have opted to remain anonymous as editors of a piece of content, will not have the ability to submit new pieces of content or validate new edits. 


\section{Content}

The knowledge database will be populated, primarily via open user submission, with brief entries explaining or documenting some single idea related to or relevant to advanced research and development fields, including mathematics, physics, chemistry, biology, statistics, computer science, programming, advanced technology and engineering, and other areas. Although the emphasis will likely at first be on content relevant to research and development in areas related to the hard sciences, the scope of the database could expand to include the social sciences, the humanities, the arts, and other disciplines, as well as more elementary content which is typically encountered at the secondary school and undergraduate levels, as appropriate.
\\
\\
All pieces of content will be of a single form containing the following items:
\begin{itemize}
\item Name --- the primary name of the idea to be displayed in searches and lists, may contain \LaTeX \textrm{ }and other special formatting; e.g. $abc$ conjecture, the $W$-trick, Equipartition theorem, Finite Field, etc.
\item Alternative Names [Aliases] (optional) --- alternate names (aliases) of the idea which may also be used for identification and search.
\item Content Type --- the type of the idea, e.g. idea, theorem, conjecture, lemma, equation, formula, inequality, code, algorithm, visualization, technique, proof, etc.
\item Text --- the actual explication/description/explanation of the idea, {\raise.17ex\hbox{$\scriptstyle\mathtt{\sim}$}}50 characters minimum\footnote{To be finalized.}, {\raise.17ex\hbox{$\scriptstyle\mathtt{\sim}$}}2000 characters maximum,\footnote{To be finalized.} can contain \LaTeX, \TeX, code, rich text, HTML, and other special formatting;\footnote{The input field will be a rich text editor containing buttons which automate much of the formatting process.} e.g. ``Given double occurrence words $w_{1}, w_{2}$, define the word distance $\delta$ by...".
\item Images (optional) --- related images, including all common file types (JPEG, PNG, TIFF, BMP, GIF, etc.).
\item Keywords / Subject Designations --- keywords and subject designations contained in or related to the idea, approximately 2 or 3 required; e.g. for a piece of content on finite fields, could include field, group, field theory, abstract algebra, associativity, operation, etc.
\item Dependencies (optional) --- references contained in the text to other pieces of content via their names or aliases, can be manually specified via special formatting and automatically generated\footnote{Possibly with user validation.} by matching words in the text  with other pieces of content in the database, will be given hyperlinks on display; e.g. ``Given {\color{blue} double occurrence words} $w_{1}, w_{2}$, define the word distance $\delta$ by...".
\item Citations (optional) --- Including both implicitly and explicitly (in-text) cited works, input with some standard academic citation format\footnote{To be specified.} (although including less information, e.g. only a url, is acceptable but discouraged), in-text citations are referenced in the text via special formatting.\footnote{To be specified.}
\end{itemize}

To prevent mass fraudulent content submissions, an authentication scheme (e.g. CAPTCHA\footnote{In the sequel, it is assumed that CAPTCHA will be used for authentication in the knowledge database.}) will be used to verify that a human is indeed making the submission.  



\subsection{Editing and Validation}

All aspects of all pieces of content will be fully editable by any user of the knowledge database. To combat malicious, inaccurate, or inappropriate content submissions and edits, an essentially democratic validation scheme will be implemented on most aspects of the content. 

\begin{itemize}
\item Name --- All modifications will be validated using the system specified below.\footnote{Which will henceforth be assumed when discussing validation, unless otherwise specified.}
\item Aliases --- Validated modifications, open (not validated) additions with CAPTCHA authorization.
\item Content Type --- Validated modifications. 
\item Text --- Validation modifications, including insertions and deletions.
\item Images --- Validated deletions and additions. 
\item Keywords / Subject Designations --- Validated modifications, open additions with CAPTCHA authorization.
\item Dependencies --- Validation of automatically (system) and manually (user) specified dependencies. 
\item Citations --- Validated modifications, open additions with CAPTCHA authorization.
\end{itemize}

In order to facilitate rapid growth and large throughput, content submissions will not be validated.\footnote{Except via later edits.} Suppose a user modifies some aspect of a piece of content which requires validation. The user's modification will not be visible until it is validated, and accepted, by the authors of the piece of content. 
%Hence the initial visibility of the user's modification will be dependent on whether their edit is of a part of the content that has been previously edited or not. If they have edited a part of the original submission, then the modification will be immediately visible since the original submission was not validated and hence we cannot assume that the submitted content is accurate and appropriate. If they have edited a part of the piece of content that was part of a previously validated modification, then the modification will not be immediately visible since validated edits are assumed to be relatively accurate and appropriate. 

The validation process will proceed as follows. After a user submits an edit, all authors of the corresponding piece of content will be signalled to the existence of the modification and will have the option of either voting for or against accepting the modification.\footnote{Or abstaining from (/ignoring) the vote.} An author may change their vote at any time prior to the closing of the vote. The edit is accepted and the voting is closed if any of the following conditions are satisfied:\footnote{Note that these conditions are evaluated with respect to the time at which they are satisfied, not with respect to the time of the edit. In particular, users who become authors after the validation process begins for some modification may still participate.}
\begin{enumerate}
\item[1.] A majority of the authors\footnote{Note that each author can vote at most once.} have voted for acceptance of the edit, or
\item[2.] if there are $N \in \mathbb{N}$ authors, at least $\lceil N/2 \rceil$ authors have voted and at least $3/4$ of the votes are for acceptance of the edit,\footnote{Note that these details are not finalized, although currently $N/2$ and $3/4$ are chosen because at least $3/4$ of the remaining $N/2$ authors would have to vote against acceptance of the edit for there to be less than a majority for acceptance. This is in general very unlikely, assuming benevolent authors.} or
\item[3.] at 5 days past modification, at least 2 votes have been submitted and at least $2/3$ of the votes are for acceptance of the edit, or
\item[4.] at 10 days past modification, at least 1 vote has been submitted and a majority of the votes are for acceptance of the edit, or
\item[5.] there are no authors,\footnote{Although it will likely be very rare, this could occur if all the authors delete their accounts.} or
\item[6.] no votes are submitted within 10 days.
\end{enumerate}
Otherwise, after 10 days, the edit is rejected and the voting process is closed. Upon acceptance of a logged in user's edit, the user\footnote{If they are not already an author.} is denoted an author of the piece of content and given the ability to participate in the edit validation process of that piece of content. In this way, there is an essentially \textit{viral authorship} system which ensures that those who validate edits have contributed useful, accurate, and appropriate edits in the past and therefore can be assumed to be relatively benevolent and have some knowledge of the content. 

The only other caveat to the editing process is that we will assign a probability of {\raise.17ex\hbox{$\scriptstyle\mathtt{\sim}$}}$1/7$\footnote{To be finalized.} to the possibility that CAPTCHA authorization will be required in order for any given user to submit an edit.\footnote{Where we assume that a user is defined by a unique IP address (or some similar metrics).} This will reduce the likelihood of a malicious user successfully using a series of computers to perform mass fraudulent modification of content in the database, while also minimizing the fixed costs of editing content for benevolent human users.

The above procedure covers all cases except when a user has submitted a piece of content that is entirely inappropriate or unsuitable for inclusion in the knowledge database or a group of authors are maliciously or inappropriately rigging the validation process in order to advance an agenda which runs counter to the principles of the knowledge database. In this case, any user will have the option to report a piece of content or a group of authors to the administrators\footnote{Likely individuals from among the organization which will maintain and develop the knowledge database.} of the database, one or more of whom will investigate and make an appropriate action. 

%Remains only to determine how deleting inappropriate content submissions should be handled

%Also, could use total transparency in submission, editing, and validation, e.g. Quora.

%Solution: "Viral" author privileges + adminstrative reviews/reports, will need (minimal) user accounts for persistance + (ideally) transparency, although anonymous (not logged in) editing is still allowed without gaining "author privileges".



\section{Application Map}

The following is a list of each of the (functional) pages\footnote{Or areas/features.} in the knowledge database and, for each page, a list of linking pages, that is, the pages that a user can navigate to from that page. Note that these pages are only defined functionally and may not correspond to separate HTML pages or any page-like interface elements. Furthermore, linked pages may not correspond to a unique button (i.e. they may be dependent on previous navigation history). This generates a corresponding ``application graph" or map, although it is too intricate to be effectively visualized. Each page (if applicable) contains a list of contained content.

\begin{enumerate}
\item[1.] Home --- 2\textsuperscript{*}, 3, 4, 6, 7, 12
\item[2.] Search Results --- 3, 4, 5\textsuperscript{*}, 6, 7, 12, 1
\begin{enumerate}
\item[] 10 content pieces matching the search criteria arranged in descending order.
\item[] For each content piece: Best matching name, content type, snippet(s) of text with matches highlighted, and all keywords.
\end{enumerate}
\item[3.] Sign-Up --- 1\textsuperscript{*}, 5
\item[4.] Login --- 1\textsuperscript{*}, 5
\item[5.] Content Page --- 8, 6, 4, 3, 2, 1, 7, 9, 11, 12, 13, 14
\begin{enumerate}
\item[] Name, alternate names, content type, text, citation, keywords, author user names, links to edits page and, if applicable, content author pages.
\end{enumerate}
\item[6.] User Page  --- 1, 2, 7, 11
\begin{enumerate}
\item[] Four separate ``tabbed" subpages: recent activity, authored content pieces, edit history, and settings. 
\item[] Recent activity: All activity on authored edits and content pieces, including acceptance/rejection of authored edits, acceptance/rejection of edits of authored content pieces, and admin report notifications. If there is a vote, a notification will urge the user to vote and link to the authored content pieces tab or directly to the edit depending on whether there is a single or multiple validating edits.
\item[] Authored content pieces: Name, content type, last modified date, notification icon indicating whether a validating edit exists or not and, if so, a ``vote" button.
\item[] Edit history: List of authored validating/accepted/rejected edits with the content piece name, content part, submission timestamp, acceptance/rejection timestamp (if applicable), and a quantitative summary of net modifications.\footnote{e.g. number of characters/words added/removed, total formatting modifications made, and images added/removed.}
\end{enumerate}
\item[7.] Content Submission Page --- 1, 5\textsuperscript{*}, 4
\item[8.] Content Editing Page --- 5\textsuperscript{*}, 1, 2, 3, 4, 6, 9, 12
\item[9.] Report Content Page --- 5\textsuperscript{*}
\item[10.] Report Authors Page --- 11\textsuperscript{*}
\item[11.] Content Authors Page\footnote{Or content validation page.} --- 5\textsuperscript{*}, 6, 1, 2, 9, 10, 12, 14
\begin{enumerate}
\item[] A list of author user names, a list of validating proposed edits and corresponding current vote results,\footnote{Still need to decide the level of information released prior to closure of the vote.}, a list of recently accepted/rejected edits with corresponding vote results, and a list of older accepted/rejected edits with links to the corresponding vote results.
\item[] For each validating edit: a ``vote" button or ``vote submitted" icon will be displayed depending on whether the user has voted or not.
\end{enumerate}
\item[12.] Admin Account Page --- 5\textsuperscript{*}, 13, 1, 2
\item[13.] Admin Action Page --- 5, 12
\item[14.] Edits Page --- 5\textsuperscript{*}, 11
\begin{enumerate}
\item[] Two lists:\footnote{Paginated with 20 edits displayed per page.} the first contains currently validating proposed edits and the second contains accepted/rejected edits.
\item[] For each validating edit: content part, creation timestamp, user name or IP address, a quantitative summary of net modifications,\footnote{e.g. number of characters/words added/removed, total formatting modifications made, and images added/removed.} and, if applicable, the edit rationale.
\item[] For each accepted/rejected edit: content part, acceptance/rejection timestamp, user name or IP address, a quantitative summary of net modifications,\footnote{See previous footnote.} and, if applicable, the edit rationale.
\end{enumerate}
\end{enumerate}

An asterisk indicates that this is the primary page to which a user will next navigate. 



\section{Use Cases}

\begin{enumerate}
\item[] Users
\begin{itemize}
\item A user must register with a username composed of at least two words separated by a space (with each word containing at least 2 characters),\footnote{That is, with their full name. These will be stored with underscores replacing the spaces.} a password containing at least 5 characters, and a unique valid email address.  
\item A user logs into their account by entering their email and password\footnote{Since their username and password might not be unique; i.e. the username is functionally only a display name.} and are then redirected to their account page.\footnote{This behavior could be more customized---a user could be redirected to their account page only if there is a pending vote on a piece of content they have authored and, otherwise, they are redirected to the home page.}
\item Upon registration, a user is sent an email containing a unique\footnote{And unguessable.} link which they must follow in order to confirm their email address. Confirmation must be completed in order to submit a piece of content or be denoted an author after having an edit accepted.\footnote{Primarily because an email is sent each time an edit is made to a piece of content the user has authored.}
\item A user must be logged in to submit a piece of content or be denoted an author after having made an accepted edit; if they attempt to submit a piece of content anonymously, they will be directed to login or sign up. 
\item When an edit is made to a piece of content which a user has authored, the user can see the existence of the edit, and be redirected to the author page of that piece of content to inspect and vote on the edit, from their account page. 
\end{itemize}
\item[] Content Retrieval
\begin{itemize}
\item Each piece of content will have its own unique url which a user can follow to view content. 
\item A user can search for content via a search bar located on all pages of the application, although the home page will be the primary entry site. Ten entries will be displayed per page in the search results. 
\end{itemize}
\item[] Content Submission
\begin{itemize}
\item After submitting a piece of content, the user will be directed to a preview page to confirm the content of their submission. Upon final submission, the user will be directed to the page of the posted piece of content. 
\item Each user will be limited to submitting at most one piece of content every minute. 
\end{itemize}
\item[] Content Editing
\begin{itemize}
\item Each user will be limited to at most one edit every 10 seconds. 
\item Upon editing, a user will have the option of including an explanation of the rationale behind the edit which will be displayed to all authors when voting.
\item After submission of an edit, all users can see the existence of that pending edit on the edits page. 
\end{itemize}
\item[] Content Validation
\begin{itemize}
\item When an edit is made to a piece of content which a user has authored, the user is sent an email notifying them of the existence of the edit and urging them to vote. Until the voting ends or they have voted, such an email is sent to the user every 4 days. 
\item When an edit is accepted or rejected, a log of it is displayed on the edits page of the associated piece of content. From the edits page, users can view the changes made or proposed by all past edits.
\item When an edit is accepted, the piece of content is updated to display the edit. 
\item While an edit is being validated, that is, a vote is open, all authors can see how many authors have voted thus far. No other information will be available until after the vote is complete, when a complete summary of the results of the vote are displayed in the authors page (including the votes of all authors). 
\end{itemize}
\item[] Admins and Violation Reports
\begin{itemize}
\item All users may submit a violation report against a piece of content, possibly calling for the deletion of the piece of content. They will only be required to give a nontrivially detailed report of the alleged violation.
\item The authors of a piece of content may also submit a violation report against another author or group of authors. They will be required to list exactly which authors are committing the alleged violation and give a nontrivially detailed report of the alleged violation.
\item When a violation report is submitted, an administrator is selected to investigate and resolve the situation. The admin may elect to take no action, send messages of guidance or warning against authors or other editing users, temporarily or permanently suspend user accounts, revoke authorship, or grant authorship\footnote{To registered users that have made some number of ``good" edits to the piece of content.} in the case\footnote{Likely rare.} where there are no authors.\footnote{Or the current authors have become essentially unactive.}\footnote{This may not be an exhaustive list of all possible admin actions.} In the event that the reporter has requested the piece of content be deleted (e.g. if the piece of content was a duplicate of another piece of content or it contained inappropriate text), then an admin may elect to immediately delete the piece of content or give authors five\footnote{Approximately.} days notice of upcoming deletion.\footnote{Primarily if the piece of content was a duplicate or was only partially unsuitable for inclusion in the knowledge database. This gives the authors time to transport some of its information over to other pieces of content if appropriate. The contents of the piece of content will also be emailed to all authors.} An admin may also elect to take action against a user who is determined to be abusing the violation reporting feature.
\end{itemize}
\end{enumerate}




\section{Technology Stack}

The following is a basic outline of the main technologies which will likely be used to implement the knowledge database. The emphasis at first will be on using technologies that are simple, proven, minimal, easy to work with, and ideally familiar.
\begin{align*}
\textrm{Database --- }& \textrm{PostgreSQL, for general storage of content and metadata.} \\
\textrm{ORM --- }& \textrm{SQLAlchemy, for interfacing with the database.} \\
\textrm{Search --- }& \textrm{Elasticsearch, for searching and matching text content.} \\ 
\textrm{Processing --- }& \textrm{Celery and RabbitMQ, for asynchronous and scheduled task processing.} \\ 
\textrm{-----------}& \textrm{--------- Custom Python API Layer --------------------} \\ \\
\textrm{Framework --- }& \textrm{Pyramid, for routing requests and serving webpages.} \\
\textrm{-----------}& \textrm{--------- REST Framework API Layer --------------------} \\ \\
\textrm{Front-End --- }& \textrm{Bootstrap / Foundation / etc. (HTML/CSS).} \\
& \textrm{Polymer / React / jQuery / etc.} \\
& \textrm{Various JavaScript libraries (\LaTeX, rich text editing, etc.).}
\end{align*}




\section{Backend Design}

The server-side backend of the knowledge database application will be composed of 3 layers, themselves containing 6 sublayers, each with varying levels of access to others. The system will essentially have the following architecture (lower on the list roughly implies closer to the client):

\begin{enumerate}
\item[1.] Database Interface Layer
\begin{enumerate}
\item[(a)] Storage API --- SQLAlchemy-based interface to the Postgres database.
\item[(b)] Search API --- ElasticSearch-based interface to the Postgres database.
\end{enumerate}
\item[2.] Main Logic Layer 
\begin{enumerate}
\item[(a)] Content API --- Submission, editing, validation, retrieval, etc. Interfaces with Storage API. 
\item[(b)] User API --- Creation, authentication, sessions, permissions, etc. Interfaces with Storage API.
\item[(c)] Admin API --- Inherits from User API, creation, authentication, etc. Interfaces with Storage API.
\end{enumerate}
\item[3.] RESTful Pyramid API --- Pyramid-based, handles all communication with web clients. Interfaces with Search API, Content API, User API, and Admin API. 
\end{enumerate}

%USING PYRAMID INSTEAD
%Might need Redis + Task Queue for sessions, votes, and violation reports...

\subsection{Storage}

A Postgres database will be the main store of all persistent data in the knowledge database application. The database will contain 11 (non-linking\footnote{There will be some additional linking tables used to implement many-to-many relationships.}) tables: Text, Content\_Piece, Content\_Type, Keyword, Name, Accepted\_Edit, Citation, User, Rejected\_Edit, Vote, and User\_Report.

\renewcommand{\arraystretch}{1.8}

\vspace{.4cm}
\begin{center}
\label{Text}
\begin{tabular}{|c|c|c|c|}
\hline
\multicolumn{4}{|c|}{Text} \\ \hline
Attribute Name & Type & Indexed & Required \\ \hline
text\_id & Integer & Yes, Primary Key & Yes \\ \hline
text & Text & No & Yes \\ \hline
timestamp & Timestamp & No & Yes \\ \hline
\multicolumn{4}{|c|}{Foreign Relationships} \\ \hline
Type & Table & \multicolumn{2}{c|}{Attribute} \\ \hline
One-to-Many & Accepted\_Edit & \multicolumn{2}{c|}{edit\_id} \\ \hline % Associated accepted edits
One-to-Many & Rejected\_Edit & \multicolumn{2}{c|}{edit\_id} \\ \hline % Associated rejected edits
\end{tabular}
\end{center}


\begin{center}
\label{ContentPiece}
\begin{tabular}{|c|c|c|c|}
\hline
\multicolumn{4}{|c|}{Content\_Piece} \\ \hline
\multicolumn{1}{|c|}{Attribute Name} & \multicolumn{1}{c|}{Type} & \multicolumn{1}{c|}{Indexed} & \multicolumn{1}{c|}{Required} \\ \hline
content\_id & Integer & Yes, Primary Key & Yes \\ \hline
%name & Text & Yes & Yes \\ \hline
timestamp & Timestamp & No & Yes \\ \hline
deleted\_timestamp & Timestamp & No & No \\ \hline
\multicolumn{4}{|c|}{Foreign Relationships} \\ \hline
Type & Table & \multicolumn{2}{c|}{Attribute} \\ \hline
One-to-One & Text & \multicolumn{2}{c|}{text\_id} \\ \hline % Text
One-to-One & Name & \multicolumn{2}{c|}{name\_id} \\ \hline	% Primary name
One-to-Many & Name & \multicolumn{2}{c|}{name\_id} \\ \hline % Alternative names
One-to-Many & Accepted\_Edit & \multicolumn{2}{c|}{name\_id} \\ \hline % All accepted edits
One-to-Many & Rejected\_Edit & \multicolumn{2}{c|}{name\_id} \\ \hline % All rejected edits
Many-to-One & Content\_Type & \multicolumn{2}{c|}{content\_type\_id} \\ \hline % Content Type
Many-to-One & User & \multicolumn{2}{c|}{user\_id} \\ \hline % Original author
Many-to-Many & Keyword & \multicolumn{2}{c|}{keyword\_id} \\ \hline	% Keywords
Many-to-Many & Citation & \multicolumn{2}{c|}{citation\_id} \\ \hline % Citations
Many-to-Many & User & \multicolumn{2}{c|}{user\_id} \\ \hline % Authors
\end{tabular}
\end{center}


\vspace*{.4cm}
\begin{center}
\label{ContentType}
\begin{tabular}{|c|c|c|c|}
\hline
\multicolumn{4}{|c|}{Content\_Type} \\ \hline
Attribute Name & Type & Indexed & Required \\ \hline
content\_type\_id & Integer & Yes, Primary Key & Yes \\ \hline
content\_type & Text & No & Yes \\ \hline
%\multicolumn{4}{|c|}{Foreign Relationships} \\ \hline
Type & Table & \multicolumn{2}{c|}{Attribute} \\ \hline
One-to-Many & Accepted\_Edit & \multicolumn{2}{c|}{edit\_id} \\ \hline % Associated accepted edits, each edit creates new citation
One-to-Many & Rejected\_Edit & \multicolumn{2}{c|}{edit\_id} \\ \hline % Associated rejected edits, each edit creates new citation
\end{tabular}
\end{center}


\vspace*{.4cm}
\begin{center}
\label{Keyword}
\begin{tabular}{|c|c|c|c|}
\hline
\multicolumn{4}{|c|}{Keyword} \\ \hline
Attribute Name & Type & Indexed & Required \\ \hline
keyword\_id & Integer & Yes, Primary Key & Yes \\ \hline
keyword & Text & Yes & Yes \\ \hline
timestamp & Timestamp & No & Yes \\ \hline
\multicolumn{4}{|c|}{Foreign Relationships} \\ \hline
%Type & Table & \multicolumn{2}{c|}{Attribute} \\ \hline
One-to-Many & Accepted\_Edit & \multicolumn{2}{c|}{edit\_id} \\ \hline % Associated accepted edits, each edit creates new keyword
One-to-Many & Rejected\_Edit & \multicolumn{2}{c|}{edit\_id} \\ \hline % Associated rejected edits, each edit creates new keyword
\end{tabular}
\end{center}


\vspace*{.4cm}
\begin{center}
\label{Name}
\begin{tabular}{|c|c|c|c|}
\hline
\multicolumn{4}{|c|}{Name} \\ \hline
Attribute Name & Type & Indexed & Required \\ \hline
name\_id & Integer & Yes, Primary Key & Yes \\ \hline
name & Text & No & Yes \\ \hline
name\_type & Text & No & Yes \\ \hline % "Primary" vs "Alternate"
timestamp & Timestamp & No & Yes \\ \hline
\multicolumn{4}{|c|}{Foreign Relationships} \\ \hline
Type & Table & \multicolumn{2}{c|}{Attribute} \\ \hline
One-to-Many & Accepted\_Edit & \multicolumn{2}{c|}{edit\_id} \\ \hline % Associated accepted edits
One-to-Many & Rejected\_Edit & \multicolumn{2}{c|}{edit\_id} \\ \hline % Associated rejected edits
\end{tabular}
\end{center}


\vspace*{.6cm}
\begin{center}
\label{AcceptedEdit}
\begin{tabular}{|c|c|c|c|}
\hline
\multicolumn{4}{|c|}{Accepted\_Edit} \\ \hline
Attribute Name & Type & Indexed & Required \\ \hline
edit\_id & Integer & Yes, Primary Key & Yes \\ \hline
redis\_edit\_id & Integer & Yes & Yes \\ \hline
edit\_text & Text & No & Yes \\ \hline
edit\_rationale & Text & No & No \\ \hline
content\_part & Text & No & Yes \\ \hline
author\_type & Text & No & Yes \\ \hline % "U" for registered users, IP address for anonymous users
timestamp & Timestamp & No & Yes \\ \hline
acc\_timestamp & Timestamp & No & Yes \\ \hline
\multicolumn{4}{|c|}{Foreign Relationships} \\ \hline
Type & Table & \multicolumn{2}{c|}{Attribute} \\ \hline
Many-to-One & User & \multicolumn{2}{c|}{user\_id} \\ \hline % Author
%One-to-One & Accepted\_Edit & \multicolumn{2}{c|}{edit\_id} \\ \hline	% Previous edit of the same content part
\end{tabular}
\end{center}


\vspace*{.4cm}
\begin{center}
\label{Citation}
\begin{tabular}{|c|c|c|c|}
\hline
\multicolumn{4}{|c|}{Citation} \\ \hline
Attribute Name & Type & Indexed & Required \\ \hline
citation\_id & Integer & Yes, Primary Key & Yes \\ \hline
citation\_text & Text & No & Yes \\ \hline
timestamp & Timestamp & No & Yes \\ \hline
\multicolumn{4}{|c|}{Foreign Relationships} \\ \hline
%Type & Table & \multicolumn{2}{c|}{Attribute} \\ \hline
One-to-Many & Accepted\_Edit & \multicolumn{2}{c|}{edit\_id} \\ \hline % Associated accepted edits, each edit creates new citation
One-to-Many & Rejected\_Edit & \multicolumn{2}{c|}{edit\_id} \\ \hline % Associated rejected edits, each edit creates new citation
\end{tabular}
\end{center}


\vspace{.6cm}
\begin{center}
\label{User}
\begin{tabular}{|c|c|c|c|}
\hline
\multicolumn{4}{|c|}{User} \\ \hline
Attribute Name & Type & Indexed & Required \\ \hline
user\_id & Integer & Yes, Primary Key & Yes \\ \hline
user\_type & Text & No & Yes \\ \hline	% "admin" or "standard"
user\_name & Text & Yes & Yes \\ \hline
email & Text & Yes & Yes \\ \hline
confirmed\_timestamp & Timestamp & No & No \\ \hline
pass\_hash & Text & Yes & Yes \\ \hline
pass\_hash\_type & Text & No & Yes \\ \hline
pass\_salt & Text & Yes & Yes \\ \hline
remember\_id & Text & Yes & No \\ \hline
remember\_token\_hash & Text & Yes & No \\ \hline
remember\_hash\_type & Text & No & No \\ \hline
timestamp & Timestamp & No & Yes \\ \hline
deleted\_timestamp & Timestamp & No & No \\ \hline
%\multicolumn{4}{|c|}{Foreign Relationships} \\ \hline
%Type & Table & \multicolumn{2}{c|}{Attribute} \\ \hline
%Many-to-One & XX & \multicolumn{2}{c|}{XX} \\ \hline
\end{tabular}
\end{center}


%\vspace*{.3cm}
\begin{center}
\label{RejectedEdit}
\begin{tabular}{|c|c|c|c|}
\hline
\multicolumn{4}{|c|}{Rejected\_Edit} \\ \hline
Attribute Name & Type & Indexed & Required \\ \hline
edit\_id & Integer & Yes, Primary Key & Yes \\ \hline
redis\_edit\_id & Integer & Yes & Yes \\ \hline
edit\_text & Text & No & Yes \\ \hline
edit\_rationale & Text & No & No \\ \hline
content\_part & Text & No & Yes \\ \hline
author\_type & Text & No & Yes \\ \hline % "U" for registered users, IP address for anonymous users
timestamp & Timestamp & No & Yes \\ \hline
rej\_timestamp & Timestamp & No & Yes \\ \hline
\multicolumn{4}{|c|}{Foreign Relationships} \\ \hline
Type & Table & \multicolumn{2}{c|}{Attribute} \\ \hline
Many-to-One & User & \multicolumn{2}{c|}{user\_id} \\ \hline % Author
\end{tabular}
\end{center}


\vspace*{.5cm}
\begin{center}
\label{Vote}
\begin{tabular}{|c|c|c|c|}
\hline
\multicolumn{4}{|c|}{Vote} \\ \hline
Attribute Name & Type & Indexed & Required \\ \hline
vote\_id & Integer & Yes, Primary Key & Yes \\ \hline
vote & Text & No & Yes \\ \hline	% "resultofvoteinstandardizedformat | user1id, Y; user2id, N; ..." Note that the result of vote summary must take into account exactly how the vote was closed (i.e. 10 days passed, all authors voted, etc.)
content\_part & Text & No & Yes \\ \hline % "content" or "authors"
timestamp & Timestamp & No & Yes \\ \hline
close\_timestamp & Timestamp & No & Yes \\ \hline	% Close of the vote
\multicolumn{4}{|c|}{Foreign Relationships} \\ \hline
Type & Table & \multicolumn{2}{c|}{Attribute} \\ \hline
One-to-One & Accepted\_Edit & \multicolumn{2}{c|}{edit\_id} \\ \hline % Edit voted on
One-to-One & Rejected\_Edit & \multicolumn{2}{c|}{edit\_id} \\ \hline % Edit voted on
Many-to-Many & User & \multicolumn{2}{c|}{user\_id} \\ \hline % voters
\end{tabular}
\end{center}


%\vspace*{.6cm}
\begin{center}
\label{UserReport}
\begin{tabular}{|c|c|c|c|}
\hline
\multicolumn{4}{|c|}{User\_Report} \\ \hline
Attribute Name & Type & Indexed & Required \\ \hline
report\_id & Integer & Yes, Primary Key & Yes \\ \hline
report\_text & Text & No & Yes \\ \hline
report\_type & Text & No & Yes \\ \hline % "content" or "authors"
author\_type & Text & No & Yes \\ \hline % "U" for registered users, IP address for anonymous users
admin\_report & Text & No & Yes \\ \hline
timestamp & Timestamp & No & Yes \\ \hline
res\_timestamp & Timestamp & No & Yes \\ \hline
\multicolumn{4}{|c|}{Foreign Relationships} \\ \hline
Type & Table & \multicolumn{2}{c|}{Attribute} \\ \hline
Many-to-One & User & \multicolumn{2}{c|}{user\_id} \\ \hline % Author
Many-to-One & User & \multicolumn{2}{c|}{user\_id} \\ \hline % Assigned admin
Many-to-One & Content\_Piece & \multicolumn{2}{c|}{content\_id} \\ \hline % Reported piece of content
\end{tabular}
\end{center}

% Add relationship to all edits on Content table, define a Vote table, review database design

% Consider desired for (validated) knowledge database vs original research portal --- actually, consider mainly how original research, represented only by newly proven theorems/lemmas, calculations, conjectures, etc., should be handled relative to the fact that the database will primarily be composed of already "known" and validated knowledge...namely, if the database allows "original" conjectures, etc., how do we determine whether they are appropriate for inclusion in the database? 


\subsubsection{Query API}
The following is a rough layout of the Storage Query API. 
\begin{enumerate}
\item[] \begin{center} \textbf{Selection Queries} \end{center}
\begin{enumerate}
\item[] Content
\begin{itemize}
\item Retrieve a piece of content.
\item Retrieve a list of alternate names.
\item Retrieve an edit and list of edits.
\item Retrieve a vote and list of votes.
\end{itemize}
\item[] Users
\begin{itemize}
\item Retrieve a user.
\item Retrieve a list of user email addresses.
\end{itemize}
\item[] Reporting
\begin{itemize}
\item Retrieve a user report and list of user reports.
\end{itemize}
\end{enumerate}
\item[] \begin{center} \textbf{Action Queries} \end{center}
\begin{enumerate}
\item[] Content
\begin{itemize}
\item Store a piece of content and associated content parts.
\item Store/update an edit, the associated content part, a vote, and the corresponding list of authors. 
\item Update a piece of content's deleted\_timestamp.
\end{itemize}
\item[] Users
\begin{itemize}
\item Store a new user.
\item Update a user's user name, email address, confirmation status, password, and remember login data.
\end{itemize}
\item[] Reporting
\begin{itemize}
\item Store a user report.
\end{itemize}
\end{enumerate}
\end{enumerate}



\subsection{Search}








\subsection{Content}


Since most application functions (submitting content pieces, searching and retrieving content pieces, etc.) expect immediate retrieval and delivery of data, most of the Content API will be synchronous. Certain functions though, most notably parts of the edit validation and voting process and the delivery of update notifications and reminders to authors, should be asynchronous and run outside of the main request-response event loop. 

Hence, Celery will be used for asynchronous task execution, with RabbitMQ selected as the broker. Redis\footnote{Configured for maximum persistance by writing every change to disk.} will be used to store results, but manually, not via Celery's result store mechanisms. Redis is chosen over the PostgreSQL database because Redis is more optimized for read and write performance.



\subsubsection{Edit Validation and Voting}

Asynchronous processing will almost exclusively\footnote{It will also be used for password recovery, at the least.} be needed for certain parts of the edit validation and voting process. The process will roughly proceed as follows (note that, where necessary, we mark events as processed synchronously or asynchronously by writing ``sync" or ``async", respectively):

\begin{enumerate}
\item[1.] An edit of a content piece is submitted and received by the Content API. The edit\footnote{And all associated metadata.} is stored in Redis (sync) and notifications are emitted to both the account pages and emails of all authors of the content piece (async). A periodic task is started which emits an additional email notification every 3 days to all authors who have not yet voted (async). Another periodic task is started which checks after 5 days and 10 days have passed whether the time-dependent conditions have been met for the vote to close (async) (if so, see 3).
\item[2.] All relevant pages check Redis for the existence of an edit each time the associated Content API functions are called. All authors of the content piece are now directed to vote whenever they visit the author page (directly, via an email link, a link from their account page, etc.). 
\item[3.] Upon submission of an author's vote, Redis (specifically the entry containing the edit) is updated with all the vote's information (sync). Calculations are then run to determine whether the vote has ended (async). If not, the process continues and no further changes are made. If the vote has ended, the edit and vote data is stored in the database and deleted from Redis, and, if the edit was accepted, the associated content piece is updated (async). Additionally, all remaining periodic tasks (from 1) are cancelled (async).
\end{enumerate}

% need to add ways for an edit to be quickly rejected
% also remember to reference the original text with each edit in case of multiple edits made in conflicting time periods



\subsubsection{Backend-linking API Layout}

The following roughly documents the layout of the backend-linking Content API; we denote class and static methods by ``class" and ``static", respectively.

\begin{enumerate}
\item[] ContentError
\item[] ApplicationError
\item[] Name
\begin{enumerate}
\item[] json\_ready (property) - JSON serializable dictionary for transmission over HTTP.
\item[] storage\_object (property)
\end{enumerate}
\item[] Text
\begin{enumerate}
\item[] json\_ready (property) - JSON serializable dictionary for transmission over HTTP.
\item[] storage\_object (property)
\end{enumerate}
\item[] UserData
\begin{enumerate}
\item[] load\_email
\item[] json\_ready (property) - JSON serializable dictionary for transmission over HTTP.
\end{enumerate}
\item[] Content - instantiation of a Content object can create a new content piece or retrieve an existing content piece. 
\begin{enumerate}
\item[] \_check\_legal (static)
\item[] \_transfer
\item[] bulk\_retrieve (class)
\item[] get\_content\_types (class)
\item[] check\_uniqueness (class)
\item[] filter\_by (class)
\item[] search (class)
\item[] autocomplete (class)
\item[] store
\item[] update (class)
\item[] \_delete
\item[] json\_ready (property) - JSON serializable dictionary for transmission over HTTP.
\end{enumerate}
\item[] DuplicateError
\item[] Edit - instantiation of an Edit object can create a new edit or retrieve an existing edit.\footnote{From Redis or the database.}
\begin{enumerate}
\item[] \_check\_legal (static)
\item[] \_retrieve\_from\_storage
\item[] \_retrieve\_from\_redis
\item[] \_transfer
\item[] bulk\_retrieve (class)
\item[] start\_vote
\item[] save
\item[] validate
\item[] \_accept
\item[] \_compute\_merging\_diff
\item[] apply\_edit
\item[] \_reject
\item[] \_notify
\item[] conflict (property)
\item[] json\_ready (property) - JSON serializable dictionary for transmission over HTTP.
\end{enumerate}
\item[] AuthorVote - instantiation of an AuthorVote object can create a new author vote or retrieve an existing author vote.
\begin{enumerate}
\item[] \_transfer
\item[] \_retrieve\_from\_storage
\item[] \_retrieve\_from\_redis
\item[] bulk\_retrieve (class)
\item[] get\_vote\_summary (class)
\item[] save
\item[] votes\_needed (class)
\item[] json\_ready (property) - JSON serializable dictionary for transmission over HTTP.
\end{enumerate}
\end{enumerate}

An email package separate from the Content API will provide email notification functionality and a diff module will provide for the calculation of edit diffs and the merging of edits with existing content parts to produce new versions. A Redis module will also provide for the storage, retrieval, and deletion of edits and votes from Redis.



\subsubsection{Client-facing API Layout}

The following roughly documents the layout of the client-facing Content View API. Whereas the backend-linking Content API provided classes and methods closely mapping application objects and their related functions, properties, and operations, the Content View API will essentially map to the ``pages" actually requested by clients; in other words, it will provide a ``view" of the backend for the client-communicating RESTful Pyramid API to consume. 

\begin{enumerate}
\item[] ContentView - instantiation creates a new content piece or retrieves an existing content piece.
\begin{enumerate}
\item[] bulk\_retrieve (class)
\item[] get\_content\_types (class)
\item[] search (class)
\item[] filter\_by (class)
\item[] autocomplete (class)
\end{enumerate}
\item[] EditView - instantiation creates a new edit or retrieves an existing edit.
\begin{enumerate}
\item[] bulk\_retrieve (class)
\item[] conflict (property)
\end{enumerate}
\item[] VoteView - instantiation creates a new vote or retrieves an existing vote.
\begin{enumerate}
\item[] bulk\_retrieve (class)
\end{enumerate}
\end{enumerate}



\end{document} 